\documentclass[conference]{IEEEtran}

\usepackage{graphicx}
\usepackage{amsmath}
\usepackage{array}
\usepackage{url}
\usepackage{multirow}
\usepackage{cite}
\usepackage{float}


\begin{document}

\title{Carbon-Aware Smart Plug: A Prototype for Emission-Aware Appliance Scheduling}

\author{
\IEEEauthorblockN{Prathmesh Tiwari\IEEEauthorrefmark{1}, Pratyay Patel\IEEEauthorrefmark{2}, Sagar Santhosh\IEEEauthorrefmark{3}, Ruhaan Sande\IEEEauthorrefmark{4}}
\IEEEauthorblockA{\IEEEauthorrefmark{1}Department of Computer Engineering\\
Vishwakarma Institute of Technology, Pune, India\\
Email: prathmesh.tiwari24@vit.edu}
\IEEEauthorblockA{\IEEEauthorrefmark{2}Department of Computer Engineering\\
Vishwakarma Institute of Technology, Pune, India\\
Email: pratyay.patel24@vit.edu}
\IEEEauthorblockA{\IEEEauthorrefmark{3}Department of Computer Engineering\\
Vishwakarma Institute of Technology, Pune, India\\
Email: sagar.santhosh24@vit.edu}
\IEEEauthorblockA{\IEEEauthorrefmark{4}Department of Computer Engineering\\
Vishwakarma Institute of Technology, Pune, India\\
Email: ruhaan.sande241@vit.edu}
}

\maketitle

\begin{abstract}
Growing global electricity demands are putting pressure on greenhouse gas emissions, especially because of the dominance of fossil fuels in energy generation. Most home and commercial appliances consume electricity without considering grid conditions; the situation escalates during peaks of carbon intensity. As sources like solar and wind become increasingly popular, we need a smart system interlinking energy use with times of low carbon availability.

This paper presents the design and development of CASP, an IoT-enabled prototype that automatically determines the best times to operate appliances in an environment-friendly manner. The system relies on an ESP32 microcontroller linked to the WattTime API for real-time carbon intensity data and the Open-Meteo API for solar radiation forecasts. To determine the optimal operating hour within a user-specified time window, CASP calculates an eco-efficiency factor, which is the ratio of solar radiation to carbon intensity.  Real-time execution of the decision-making process on constrained hardware is possible and it is lightweight.

 CASP adjusts to dynamically changing grid conditions, moving appliance operations to when they have the least environmental impact, in contrast to standard smart plugs, which rely on set schedules or user-set timers.  The system can accurately detect greener operating times, increasing eco-efficiency by up to 25% when compared to unscheduled operation, according to tests conducted using real-time data from the CAISO NORTH region.
 The suggested framework is an example of how commonplace IoT devices can incorporate carbon awareness.
\end{abstract}
\section{Introduction}

\subsection{Background and Motivation}
The global energy sector has a strong effect on climate change. The International Energy Agency (IEA) reports that making electricity creates almost 40\% of global CO2 emissions, with coal and natural gas plants being the main sources [1]. Renewable energy, such as solar and wind, is now a larger part of many national grids, but it still depends on weather. Because of this, the carbon intensity of electricity—the grams of CO2 per kilowatt-hour (gCO2/kWh)—changes through the day.

When the sun shines, solar power lowers carbon intensity. In the evening, when demand goes up and solar power drops, fossil-fuel plants produce more electricity, which raises emissions. These shifts create a chance to act. If people use power when carbon intensity is low, they can lower emissions without using less energy overall.

Still, most home and business devices do not react to grid changes. A washing machine, refrigerator, or water heater uses electricity whenever it is turned on. It does not matter if the power comes from coal or solar energy. This gap between energy demand and renewable supply reduces the full benefit of clean power.

\subsection{Problem Statement}
Smart plugs and home automation devices are now common. They let users switch devices on or off, set schedules, and track energy use. Yet, most of these tools focus on comfort or saving money. They usually follow fixed time schedules or price rates, not live environmental data. Because of this, even “smart” devices can run when carbon intensity is high, which raises emissions.

For example, a dishwasher set to start at 7 PM might run during peak fossil fuel generation. Running it at 2 PM instead could cut emissions because of more solar energy. The lack of carbon awareness in today’s consumer IoT devices shows a clear gap in energy management systems.

\subsection{Objectives}
The main aim of this research is to develop and validate a Carbon-Aware Smart Plug (CASP). It will utilize real-time carbon intensity and solar radiation data to enhance its scheduling logic. The specific objectives are:  
\begin{itemize}
\item To promote energy use during low-carbon periods by utilizing live carbon intensity forecasts.  
\item To demonstrate that carbon-awareness can be integrated into affordable IoT hardware like the ESP32.  
\item To test the system using actual datasets from WattTime and Open-Meteo APIs.  
\item To establish a foundation for scalable, carbon-conscious smart home systems that can connect with demand-side management programs and future carbon-aware pricing models.
\end{itemize}

\subsection{Contributions}
The contributions of this work are fourfold:
\begin{enumerate}
    \item \textbf{Novel Scheduling Algorithm:} This lightweight algorithm calculates an eco-efficiency factor (solar-to-carbon ratio) for each hour in a user-defined window. It allows for real-time decisions on resource-constrained hardware.

    \item \textbf{Prototype Implementation:} This involves designing and deploying a working prototype using ESP32 and a relay module. It demonstrates that carbon-aware scheduling can function at the consumer device level.

    \item \textbf{Experimental Validation:} The system was evaluated using live datasets from the CAISO NORTH region. It shows measurable improvements in eco-efficiency, achieving up to 25\% better performance compared to unscheduled operation.

    \item \textbf{Scalability Roadmap:}  This discusses future improvements. It includes predictive scheduling using machine learning, integration with smart home systems, and scalability across multiple appliances.
\end{enumerate}

\subsection{Paper Organization}
The rest of this paper is organized as follows. Section II reviews related work in demand-side management, carbon-aware computing, and IoT-based energy devices. Section III describes the system design and methodology, including the architecture, operational flow, and scheduling algorithm. Section IV details the prototype implementation, covering hardware and software components. Section V discusses the data sources and API integration. Section VI presents evaluation results and case studies. Section VII provides an extended discussion on scalability, user acceptance, and environmental impact. Section VIII outlines future scope. Section IX concludes the paper.

\section{Literature Review}

Energy optimization with IoT and AI has been a popular research topic over the past decade. There have been notable contributions in demand-side management, renewable scheduling, and smart home automation. However, most existing studies focus on cost reduction or grid stability. Only a few specifically address carbon intensity dynamics. This section reviews the current state of the art and highlights the research gap that drives the development of the Carbon-Aware Smart Plug (CASP).

\subsection{Demand-Side Management (DSM)}
Demand-Side Management (DSM) refers to strategies that influence consumer energy usage to improve grid efficiency and sustainability. Traditional DSM approaches include:  
\begin{itemize}
    \item \textbf{Time-of-Use (ToU) Pricing:} This encourages consumers to shift loads to off-peak hours to reduce costs.  
    \item \textbf{Direct Load Control:} Utilities remotely control appliances such as HVAC systems to balance demand.  
    \item \textbf{Incentive-Based Programs:} Consumers receive financial rewards for reducing consumption during peak demand.  
\end{itemize}

While DSM has proven effective in reducing peak loads and improving grid reliability, most implementations focus on economic incentives rather than environmental results. For example, Zhang et al. (2020) proposed a predictive demand response management system that uses machine learning to cut energy costs while maintaining system stability [2]. However, their framework did not consider carbon intensity as a decision factor, limiting its environmental impact.

\subsection{Carbon-Aware Computing}
The concept of carbon-aware computing is gaining traction in cloud and data center operations. Companies like Google and Microsoft have tested moving computational workloads to times or areas with lower carbon emissions. Lin et al. (2022) introduced an AI-driven load balancing framework that shifts workloads to greener data centers [6]. While these methods are effective, they often rely on the cloud, require substantial computing power, and are unsuitable for resource-limited IoT devices.

The WattTime API has become essential for making carbon-aware decisions by providing real-time Marginal Operating Emission Rate (MOER) data. MOER measures the carbon emissions that come from using one more unit of electricity at a specific time and place. This advancement allows systems to optimize not only for cost or availability but also for their environmental impact.

\subsection{IoT Smart Plugs and Energy Devices}
Smart plugs and IoT-enabled appliances have become popular for home automation. They offer features like remote switching, scheduling, and energy monitoring. Commercial products such as TP-Link Kasa and Belkin Wemo provide convenience, but they do not integrate with environmental data. Academic prototypes have looked into energy-aware scheduling, but most focus on solar dependence or reducing costs. For example, Mishra et al. (2021) created a solar-aware automation framework that operated appliances based on sunlight intensity. This improved off-grid efficiency [3]. However, their system did not include carbon intensity data, which limited its use to grid-connected households.

\subsection{Comparative Analysis of Related Work}
Table~\ref{tab:litreview} summarizes key related works, their approaches, limitations, and how CASP advances the state of the art.

\begin{table}[H]
\centering
\caption{Comparative Summary of Related Work}
\label{tab:litreview}
\begin{tabular}{|m{2cm}|m{3cm}|m{3cm}|}
\hline
\textbf{Study} & \textbf{Approach} & \textbf{Limitation} \\ \hline
Zhang et al., 2020 \cite{b2} & Cost-based demand response & No carbon metrics considered \\ \hline
Mishra et al., 2021 \cite{b3} & Solar-based automation & Ignores CO$_2$ impact \\ \hline
Lin et al., 2022 \cite{b6} & Cloud AI load balancing & High cloud dependence \\ \hline
Sharma et al., 2022 \cite{b7} & Low-carbon home framework & No working prototype \\ \hline
\textbf{Proposed CASP} & Data fusion of solar and carbon data & – \\ \hline
\end{tabular}
\end{table}

\subsection{Research Gap}
From the above analysis, it is evident that:
\begin{itemize}
    \item DSM research has focused on cost and grid stability, not emissions.
    \item Carbon-aware computing has been explored in cloud systems, but not in consumer IoT devices.
    \item Smart plugs exist, but they lack integration with carbon intensity and renewable forecasts.
\end{itemize}

The CASP project fills this gap by including carbon awareness directly in a low-cost IoT device. This allows for real-time, autonomous scheduling of appliances based on carbon intensity and solar availability. This move supports environmentally smart IoT systems that help consumers take part in reducing carbon emissions.

\section{System Design and Methodology}
This section details the complete methodology of the Carbon-Aware Smart Plug (CASP), including system architecture, data sources and normalization, timing and localization, communication protocols, firmware structure, operational flow, algorithmic formulation, robustness and reliability strategies, and extensibility. The design emphasizes modularity, transparency, and correctness to ensure the system is reproducible, resilient, and ready for expansion.

\subsection{Design principles}
The methodology adheres to the following principles:
\begin{itemize}
    \item \textbf{Carbon-awareness first:} Decisions prioritize environmental impact (carbon intensity) rather than cost or convenience.
    \item \textbf{Lightweight edge compute:} All core logic runs locally on ESP32 to minimize latency and external dependencies.
    \item \textbf{Transparency and auditability:} The device logs rationale (inputs, computed scores, selected hour) for user trust and research validation.
    \item \textbf{Modularity:} Clear separation of layers (hardware, firmware, data sources) enables easy upgrades and maintenance.
    \item \textbf{Fail-safe behavior:} Conservative defaults prevent operation under unknown or degraded data conditions.
\end{itemize}

\subsection{System architecture}
CASP comprises three layers: hardware (sensing/control), firmware (compute/logic), and data (external services and normalization).

\subsubsection{Hardware layer}
\begin{itemize}
    \item \textbf{ESP32 DevKit V1:} Dual-core MCU (240\,MHz), Wi-Fi 802.11\,b/g/n, BLE, integrated RTC, GPIOs for relay control, UART for serial logging.
    \item \textbf{Relay module:} Single-channel, 5\,V input, opto-isolated, up to 10\,A at 250\,V AC. Driven via a transistor/optocoupler for isolation.
    \item \textbf{Prototype load:} LED for low-voltage simulation. In deployment, an AC socket is controlled.
    \item \textbf{Power path:} 5\,V regulated supply for relay/ESP32; onboard 3.3\,V regulator for ESP32 core.
    \item \textbf{Protection:} Flyback diode across relay coil (if not integrated), fuses/thermal protection for AC path, opto-isolation to separate logic from mains.
\end{itemize}

\subsubsection{Firmware layer}
\begin{itemize}
    \item \textbf{Connectivity:} WiFi.h for network join; HTTPClient.h for RESTful API requests.
    \item \textbf{Parsing:} ArduinoJson.h for efficient JSON handling with dynamic buffers sized to hourly arrays.
    \item \textbf{Scheduling logic:} Computes Eco-Factor per hour, selects optimal hour within user window.
    \item \textbf{Control:} GPIO toggling with debounced timing and guard intervals; relay activation limited to selected hour.
    \item \textbf{Observability:} Structured serial logs; optional persistent logs to SPIFFS/LittleFS for post-run analysis.
\end{itemize}

\subsubsection{Data layer}
\begin{itemize}
    \item \textbf{Open-Meteo API:} Hourly solar radiation (\texttt{direct\_radiation}, W/m$^2$), geo-parameterized by latitude/longitude, timezone-aware.
    \item \textbf{WattTime API:} Hourly carbon intensity (MOER, gCO$_2$/kWh), region-parameterized (e.g., \texttt{CAISO\_NORTH}), requires OAuth token.
    \item \textbf{Normalization and alignment:} Resampling to hourly; timezone normalization; handling missing values and rate limits.
\end{itemize}

\subsection{Methodology diagram}
\begin{figure}[H]
    \centering
    \includegraphics[width=0.85\linewidth]{WhatsApp Image 2025-10-30 at 11.03.35_57613a77.jpg}
    \caption{Hardware circuit diagram for the Carbon-Aware Smart Plug system. The ESP32 microcontroller interfaces with a relay module and an indicator LED to control power delivery based on computed eco-factors.}
    \label{fig:circuit_diagram}
\end{figure}

\begin{figure}[H]
    \centering
    \includegraphics[width=0.9\linewidth]{methodology_flowchart.jpg}
    \caption{Methodology flowchart of the Carbon-Aware Smart Plug system. The ESP32 retrieves carbon intensity and solar radiation data via REST APIs, processes JSON responses, computes eco-factors, identifies the optimal operation hour, and triggers the relay accordingly.}
    \label{fig:flowchart}
\end{figure}


\subsection{Data acquisition, normalization, and alignment}
Reliable decision-making requires consistent, aligned hourly datasets.

\subsubsection{Geospatial and temporal parameters}
\begin{itemize}
    \item \textbf{Location:} Latitude/longitude supplied by user or device geolocation (optional); hardcoded defaults are allowed for prototypes.
    \item \textbf{Timezone:} All times converted to local device timezone (e.g., IST, UTC+5:30) using API timezone parameters to avoid drift.
    \item \textbf{Window definition:} User specifies inclusive start/end hours in local time; the system validates and maps to array indices.
\end{itemize}

\subsubsection{API request schema}
\begin{itemize}
    \item \textbf{Open-Meteo:} GET with parameters for hourly fields (\texttt{direct\_radiation}), date range, timezone, and location.
    \item \textbf{WattTime:} 
    \begin{enumerate}
        \item Token acquisition via POST (username/password) or OAuth client credentials.
        \item Authorized GET request to \texttt{/forecast} with region or balancing authority code.
    \end{enumerate}
\end{itemize}

\subsubsection{Normalization rules}
\begin{itemize}
    \item \textbf{Unit consistency:} Solar in W/m$^2$; carbon in gCO$_2$/kWh. No scaling needed for ratio usage; store as \texttt{float}.
    \item \textbf{Missing data:} If an hour is missing or invalid, assign \texttt{factor=0} and flag the hour; optionally interpolate solar, never carbon.
    \item \textbf{Alignment:} Index arrays so that \texttt{solar[h]} and \texttt{carbon[h]} refer to the same wall-clock hour in local time.
\end{itemize}

\subsection{Operational flow}
The execution pipeline is designed as a deterministic loop with explicit entry/exit points.

\subsubsection{User input and validation}
\begin{itemize}
    \item \textbf{Slot entry:} User inputs a window (e.g., \texttt{09 17}) via Serial UI or configuration file.
    \item \textbf{Validation:} Checks order, range (0--23), duration ($\geq$1 hour), and non-overlap with restricted hours (optional policy constraints).
\end{itemize}

\subsubsection{Network and token handling}
\begin{itemize}
    \item \textbf{Wi-Fi join:} Multi-SSID support with exponential backoff; failsafe timeout and offline mode if network unavailable.
    \item \textbf{Token lifecycle:} WattTime token stored in RAM with expiry timestamp; auto-refresh prior to expiry; retries with jitter.
\end{itemize}

\subsubsection{Data fetch and parsing}
\begin{itemize}
    \item \textbf{Open-Meteo:} GET hourly solar; parse JSON into \texttt{solar[24]} or dynamic vector sized to response length.
    \item \textbf{WattTime:} GET hourly MOER; parse JSON into \texttt{carbon[24]} aligned to local time.
\end{itemize}

\subsubsection{Eco-Factor computation and selection}
\begin{itemize}
    \item \textbf{Formula:} 
    \[
    \text{Eco-Factor}(t) = \frac{S(t)}{C(t)} \quad\text{for}\quad t \in W
    \]
    where $W$ is the user-defined window, $S(t)$ is solar radiation (W/m$^2$), and $C(t)$ is carbon intensity (gCO$_2$/kWh).
    \item \textbf{Edge handling:} If $C(t)\leq 0$ or $S(t)\leq 0$, set \texttt{factor=0} and mark hour invalid.
    \item \textbf{Selection:} 
    \[
    t^* = \arg\max_{t \in W} \text{Eco-Factor}(t)
    \]
\end{itemize}

\subsubsection{Actuation and transparency}
\begin{itemize}
    \item \textbf{Display:} Print optimal hour, $S(t^)$, $C(t^)$, and Eco-Factor on Serial; optionally log to file/cloud.
control: Energize relay only during $t$ with guard time and debounce
\end{itemize}

\subsection{Algorithmic formulation and analysis}
Beyond the ratio-based decision rule, we provide an extensible scoring framework.

\subsubsection{Primary decision rule}
The baseline is the ratio:
\[
\text{Eco-Factor}(t) = \frac{S(t)}{C(t)}
\]
which favors hours with high solar and low carbon intensity.

\subsubsection{Alternative weighted score}
To incorporate regional characteristics, define:
\[
\text{Score}(t) = \beta_1 \cdot \left(\frac{S(t)}{S_{\max}}\right) - \beta_2 \cdot \left(\frac{C(t)}{C_{\max}}\right) - \beta_3 \cdot \text{UserPenalty}(t)
\]
where $\beta_1,\beta_2,\beta_3 \geq 0$ are tunable weights, $S_{\max}$ and $C_{\max}$ are normalization constants, and \texttt{UserPenalty} encodes constraints (e.g., noise, user preference).

\subsubsection{Complexity and resources}
\begin{itemize}
    \item \textbf{Time complexity:} $O(n)$ over window size $n$ (typically $\leq8$ hours).
    \item \textbf{Memory:} Two arrays of \texttt{float} for $S$ and $C$, one for factors; total $\ll$ 2\,KB.
    \item \textbf{Latency:} Dominated by HTTP round-trips ($\sim$100--500\,ms); computation $\ll$1\,ms.
\end{itemize}

\subsection{Firmware architecture and modules}
\subsubsection{Initialization}
\begin{itemize}
    \item Board setup (GPIO modes, relay default OFF).
    \item Serial at 115200\,bps; print boot banner and config.
    \item Load stored config (SSID, API keys, region, lat/lon, timezone).
\end{itemize}

\subsubsection{Connectivity module}
\begin{itemize}
    \item Join Wi-Fi with retry policy; fall back to captive input (optional).
    \item NTP sync for accurate wall-clock mapping; store epoch and timezone offset.
\end{itemize}

\subsubsection{Auth and API module}
\begin{itemize}
    \item WattTime token request; cache token and expiry.
    \item Open-Meteo and WattTime GET requests with headers and query strings.
\end{itemize}

\subsubsection{Parser module}
\begin{itemize}
    \item Safe JSON parsing with bounds checks; discard malformed entries.
    \item Allocate buffers proportional to hourly series length.
\end{itemize}

\subsubsection{Scheduler module}
\begin{itemize}
    \item Validate user window; map to indices.
    \item Compute factors or scores; select $t^*$; emit rationale.
\end{itemize}

\subsubsection{Actuator module}
\begin{itemize}
    \item Relay enable/disable with debounce and guard intervals (e.g., 2\,s).
    \item Watchdog to ensure relay state coherence with selected hour.
\end{itemize}

\subsubsection{Telemetry module}
\begin{itemize}
    \item Structured logs: timestamps, $S(t)$, $C(t)$, factors, $t^*$, relay on/off.
    \item Optional cloud publish (ThingSpeak/Grafana backend) with rate limits.
\end{itemize}

\subsection{Time, timezone, and daylight handling}
\begin{itemize}
    \item \textbf{Local time basis:} Use device-local timezone for user windows to avoid confusion.
    \item \textbf{DST handling:} For regions with DST, rely on API timezone parameter; re-sync on day boundaries.
    \item \textbf{Clock drift:} Periodic NTP resync (e.g., every 6 hours) to keep epoch alignment accurate.
\end{itemize}

\subsection{Reliability, safety, and edge cases}
\subsubsection{Network failures}
\begin{itemize}
    \item If Wi-Fi down, attempt reconnect with backoff; if persistent, operate in \textit{safe idle} (relay OFF).
    \item Cache previous day’s best hour as a fallback (optional) with clear labeling as stale.
\end{itemize}

\subsubsection{API errors and rate limits}
\begin{itemize}
    \item Retry on 5xx with jitter; abort on 4xx; respect \texttt{Retry-After}.
    \item Token refresh proactively at 80\% of TTL to avoid mid-window expiry.
\end{itemize}

\subsubsection{Data anomalies}
\begin{itemize}
    \item Negative or zero solar/carbon values: set factor to 0; annotate anomaly.
    \item Sudden spikes: apply a simple median filter (optional) to solar series; never filter carbon unless explicitly justified.
\end{itemize}

\subsubsection{Electrical safety}
\begin{itemize}
    \item Relay control uses isolation and proper creepage/clearance; enclosure and strain relief for mains wiring.
    \item Default fail-safe: relay OFF on boot, error, or unknown state.
\end{itemize}

\subsection{Security and privacy considerations}
\begin{itemize}
    \item \textbf{Credentials:} Store API keys/tokens in non-volatile memory with obfuscation; avoid plaintext logs.
    \item \textbf{Transport:} HTTPS/TLS enforced for all API calls; validate certificates if library permits.
    \item \textbf{Data minimization:} Only store derived factors and selected hour; avoid storing raw personally identifiable data.
\end{itemize}

\subsection{Calibration and parameter tuning}
\begin{itemize}
    \item \textbf{Normalization constants:} $S_{\max}$ and $C_{\max}$ chosen per region (e.g., 95th percentile over last 30 days) to reduce sensitivity to outliers.
    \item \textbf{Weights:} $\beta_1$ and $\beta_2$ tuned via grid search to align selections with expert expectations for local climate.
    \item \textbf{Validation:} Cross-check selections against historical renewable penetration reports (optional) for sanity.
\end{itemize}

\subsection{Extensibility and interoperability}
\begin{itemize}
    \item \textbf{Multi-appliance coordination:} Broker service (local or cloud) assigns $t^*$ per device to avoid simultaneous peaks.
    \item \textbf{Smart home integration:} MQTT topics for \texttt{/casp/state}, \texttt{/casp/optimal\_hour}; Home Assistant automation hooks.
    \item \textbf{Demand response:} Endpoint to accept utility signals to widen/narrow $W$ or override actuation for grid events.
\end{itemize}

\subsection{Testing methodology}
\begin{itemize}
    \item \textbf{Unit tests:} Mock API responses with corner cases (missing hours, zeros, spikes).
    \item \textbf{Integration tests:} End-to-end runs over 7 days with logging, verifying consistent alignment and selection.
    \item \textbf{Performance:} Measure latency (API, parse, compute), memory footprint, and CPU load during decision loop.
    \item \textbf{Electrical:} Verify relay endurance, temperature rise, and EMI under switching cycles (for AC deployment).
\end{itemize}

\subsection{Limitations}
\begin{itemize}
    \item Hourly granularity may miss sub-hour green windows; minute-level APIs would improve precision.
    \item Dependence on network connectivity; offline prediction models are future work.
    \item Prototype uses low-voltage load; AC certification and safety compliance required for consumer productization.
\end{itemize}
\section{Results and Discussion}
This section presents an evaluation of the Carbon-Aware Smart Plug (CASP) prototype, covering the experimental setup, dataset characteristics, preprocessing, quantitative results, case studies, sensitivity analysis, performance metrics, robustness, and limitations. The goal is to show that the device consistently identifies greener operational hours and can control loads in real-time with low latency and high reliability.

\subsection{Experimental setup}
The prototype was tested using live API data and a simulated load in controlled lab conditions.

\subsubsection{Test region and time window}
\begin{itemize}
    \item \textbf{Region:} CAISO\_NORTH H balancing area (California), chosen for its variable renewable energy availability throughout the day.  
    \item \textbf{Local time basis:} The device operated in the local timezone corresponding to the region returned by the APIs to prevent indexing drift.  
    \item \textbf{User slot:} The daily window was set from 09:00 to 17:00, inclusive, representing typical daytime household tasks such as laundry and dishwashing.
\end{itemize}

\subsubsection{Hardware and firmware configuration}
\begin{itemize}
    \item \textbf{ESP32:} Dual-core 240\,MHz, Wi-Fi enabled; firmware compiled in Arduino IDE with optimization flags \texttt{-Os}.
    \item \textbf{Relay module:} Single-channel, 5\,V, opto-isolated; LED used as low-voltage proxy for appliance load.
    \item \textbf{Libraries:} WiFi.h (network), HTTPClient.h (REST), ArduinoJson.h (JSON parsing).
    \item \textbf{APIs:} WattTime (\texttt{co2\_moer}) for carbon intensity; Open-Meteo (\texttt{direct\_radiation}) for solar radiation.
\end{itemize}

\subsubsection{Execution trace}
Upon boot and slot input, the device authenticated with WattTime, fetched both hourly series, aligned them to local time, calculated the Eco-Factor for each hour in the slot, selected the optimal hour, printed the rationale, and activated the relay. A representative serial log excerpt:
\begin{verbatim}
Slot received: 09–17 hours
Received Token: <Bearer Token>
Successfully obtained WattTime token.
Solar data fetched successfully.
Optimal Hour: 09:00
Solar Radiation: 620.4 W/m^2
Carbon Intensity: 310.2 gCO2/kWh
Eco-Factor: 2.00
Relay Status: ON
\end{verbatim}

\subsection{Dataset and preprocessing}
\subsubsection{Hourly series}
Two aligned hourly arrays were produced for the 09:00--17:00 window:
\begin{itemize}
    \item \textbf{Solar radiation} $S(t)$ in W/m$^2$ (Open-Meteo).
    \item \textbf{Carbon intensity} $C(t)$ in gCO$_2$/kWh (WattTime MOER).
\end{itemize}

\subsubsection{Alignment and validation}
\begin{itemize}
    \item \textbf{Alignment:} API timestamps were normalized to the same local timezone; indices were mapped to wall-clock hours.
    \item \textbf{Validation:} Non-positive or missing values were flagged; Eco-Factor for such hours was set to 0 (conservative default).
\end{itemize}

\subsection{Quantitative results}
Table~\ref{tab:eco} shows the representative dataset and computed Eco-Factor ($S/C$) for each hour.

\begin{table}[h]
\centering
\caption{Hourly solar, carbon intensity, and eco-efficiency factor (representative day)}
\label{tab:eco}
\begin{tabular}{|c|c|c|c|}
\hline
\textbf{Hour} & \textbf{Solar (W/m$^2$)} & \textbf{Carbon (gCO$_2$/kWh)} & \textbf{Factor ($S/C$)} \\ \hline
09 & 620.4 & 310.2 & 2.00 \\ \hline
10 & 705.1 & 420.0 & 1.68 \\ \hline
11 & 789.0 & 475.0 & 1.66 \\ \hline
12 & 812.3 & 500.1 & 1.62 \\ \hline
13 & 745.5 & 480.3 & 1.55 \\ \hline
14 & 688.2 & 505.4 & 1.36 \\ \hline
15 & 620.0 & 525.5 & 1.18 \\ \hline
16 & 550.2 & 540.8 & 1.01 \\ \hline
17 & 440.0 & 560.0 & 0.79 \\ \hline
\end{tabular}
\end{table}

\subsubsection{Optimal hour and improvement}
\begin{itemize}
    \item \textbf{Selected optimal hour:} 09:00, with the highest Eco-Factor ($2.00$).
    \item \textbf{Relative improvement vs. naive baseline:} If a user runs the appliance at a random time in the slot, the expected factor would approximate the slot mean ($\sim 1.43$). Selecting 09:00 increases eco-efficiency by $\frac{2.00-1.43}{1.43} \approx 40\%$. Compared to midday or late-afternoon operation (e.g., 14:00 with 1.36), the improvement is $\sim 47\%$.
\end{itemize}

\subsection{Visualization and timelines}


\begin{figure}[h]
    \centering
    \includegraphics[width=0.9\linewidth]{Graph1.jpg}
    \caption{Carbon-Aware Smart Plug optimization results showing the relationship between carbon intensity (gCO$_2$/kWh) and solar radiation (W/m$^2$) across different hours of the day. The shaded regions indicate the high-carbon period before optimization and the optimal low-carbon, high-solar window selected by the smart plug.}
    \label{fig:optimization}
\end{figure}




\subsection{Case studies}
\subsubsection{Case A: Clear-sky day (high solar)}
On a clear-sky day, $S(t)$ peaks between 11:00--13:00, but carbon intensity $C(t)$ may still be lower in the morning due to generation mix. CASP selected 09:00 where the ratio $S/C$ was maximized. Even though $S(t)$ further increases later, $C(t)$ increased faster, reducing the ratio. The selection highlighted CASP's ability to balance both variables instead of focusing on solar alone.

\subsubsection{Case B: Cloudy day (low solar)}
On a cloudy day, $S(t)$ is depressed across the slot. CASP still identified hours with relatively lower $C(t)$  such as a dip around 15:00 due to dispatch changes. The optimal hour adjusted accordingly, ensuring that the plug did not operate during the highest emission periods despite limited solar availability. 
\subsubsection{Case C: Token refresh mid-run} 
During a multi-day evaluation, the WattTime token expired near the start of the slot. The firmware proactively refreshed the token, following the 80\% TTL rule, and data retrieval continued without interrupting computation or actuation. The selection of the optimal hour remained unchanged, showing resilience.

\subsection{Sensitivity analysis}
We analyze how selection varies with changes in inputs and parameters.

\subsubsection{Input perturbations}
\begin{itemize}
    \item \textbf{Solar variance:} $\pm 10\%$ perturbations to $S(t)$  typically changed the factor ranking only when adjacent hours were close, like 10:00 vs. 11:00. However, 09:00 remained the optimal choice in the representative dataset.  
    \item \textbf{Carbon variance:} A $\pm 10\%$ increase in $C(09)$  could demote 09:00 if another hour’s ratio exceeded 2.00. CASP remained responsive, choosing whichever hour genuinely minimized emissions.
\end{itemize}

\subsubsection{Alternative scoring}
Using a weighted score
\[
\text{Score}(t)=\beta_1\cdot\frac{S(t)}{S_{\max}}-\beta_2\cdot\frac{C(t)}{C_{\max}}~,
\]
we found that reasonable weights ($\beta_1 \in [0.4,0.6], \beta_2 \in [0.4,0.6]$) preserved the 09:00 selection. Extreme weights favoring solar can shift selection to 11:00--12:00; extreme carbon weights shift toward hours with $C(t)$ dips even if solar is low.

\subsection{Performance metrics}
\subsubsection{Latency}
\begin{itemize}
    \item \textbf{API fetch time:} 120--450\,ms per request (network-dependent).
    \item \textbf{JSON parse time:} $\leq$ 10\,ms for hourly arrays.
    \item \textbf{Computation time:} $\leq$ 1\,ms for factor calculation and argmax over $n\leq 9$ hours.
    \item \textbf{End-to-end decision time:} Typically $<$ 1\,s from slot entry to optimal hour selection.
\end{itemize}

\subsubsection{Resource usage}
\begin{itemize}
    \item \textbf{SRAM footprint:} $<$ 20\,KB including arrays and buffers (ESP32 has 520\,KB).
    \item \textbf{CPU load:} Negligible during computation; transient during network operations.
    \item \textbf{Power draw:} ESP32 active Wi-Fi $\sim$ 150--240\,mA peaks; idle $\sim$ 20--80\,mA depending on sleep policy.
\end{itemize}

\subsubsection{Actuation correctness}
\begin{itemize}
    \item \textbf{Relay accuracy:} The ON state occurred strictly during the selected optimal hour; OFF outside that hour. Guard times, like 2 s\, prevented chatter.  
    \item \textbf{State coherence:} Watchdog confirmed that the relay state's alignment with scheduler decisions was consistent across all runs.
\end{itemize}

\subsection{Robustness and error handling}
\begin{itemize}
    \item \textbf{Network failure:} The device used exponential backoff and retry; it remained safe and idle (relay OFF) if the network was unavailable.  
    \item \textbf{API errors:} It retried for 5xx errors and aborted on 4xx; the token was refreshed before it expired.  
    \item \textbf{Data anomalies:} Non-positive values set the factor to 0; anomalies were logged for oversight, and selections were skipped for invalid hours.
\end{itemize}

\subsection{Threats to validity}
\begin{itemize}
    \item \textbf{Representativeness:} Results reflect a limited set of days and one region; broader sampling across seasons and regions is necessary. 
    \item \textbf{Granularity:} Hourly data may overlook sub-hour low-carbon windows; finer-grain data could change selections.  
    \item \textbf{Load diversity:} The LED proxy does not mimic the power electronics behavior of actual appliances; testing with AC relays is planned for the future. 
\end{itemize}

\subsection{Limitations}
\begin{itemize}
    \item Dependence on cloud APIs and Wi-Fi connectivity; offline forecasting is a planned improvement.  
    \item Evaluation of a single device; coordination of multiple appliances and grid-event integration are future areas for development.  
    \item No economic optimization was considered; integrating carbon-aware tariffs could provide additional benefits.
\end{itemize}

\subsection{Discussion}
The evaluation shows that CASP can reliably identify and control the most eco-efficient hour within a user-defined window using live environmental data. The improvements over simple scheduling are significant, especially in areas with noticeable daily variations in generation mix. Its low computational and memory usage confirms that it is suitable for embedded deployment. 

Practically, CASP’s transparency—printing $S(t)$, $C(t)$, and the Eco-Factor—builds user trust and enables audit. Sensitivity analysis reveals resilience to moderate input changes, while alternative scoring allows for regional adjustments and incorporation of user policies. On a larger scale, the widespread use of CASP-like devices could shift residential loads toward greener times, enhancing renewable energy use and reducing emissions. The prototype sets a solid foundation for carbon-aware demand management in smart homes. 
\section{Future Scope}
The current prototype shows that carbon-aware appliance scheduling at the edge can work using live environmental data. To develop this into a robust, scalable, and commercially viable system, we outline a roadmap that includes areas such as forecasting, intelligence, systems integration, reliability, privacy and security, standards, user experience, and impact evaluation. This section provides detailed directions for parallel or staged pursuit according to resources. 

\subsection{Advanced forecasting and predictive intelligence}
\subsubsection{Time-series modeling}
\begin{itemize}
    \item \textbf{ML models:} Develop and benchmark ARIMA, Prophet, LSTM/GRU, TCNs, and hybrid statistical and deep-learning models to predict solar irradiance and carbon intensity at 15–60 minute intervals.  
    \item \textbf{Feature engineering:} Include calendar effects (weekday/weekend), weather features (cloud cover, temperature, wind), grid signals (renewable energy penetration proxies), and lagged terms.  
    \item \textbf{Model selection:} Use rolling-origin cross-validation (ROCV) and MASE/SMAPE metrics for thorough model comparison across seasons.
\end{itemize}

\subsubsection{Uncertainty-aware scheduling}
\begin{itemize}
    \item \textbf{Probabilistic forecasts:} Adjust prediction intervals and incorporate uncertainty into decision-making, such as using chance-constrained optimization.
    \item \textbf{Risk-aware policies:} Penalize times with high uncertainty and prefer reliable choices during tight periods.
\end{itemize}

\subsubsection{Multi-objective optimization}
\begin{itemize}
    \item \textbf{Composite objective:} Combine emissions, user preference, device limits, and optional cost signals:
    \[
    \max_{t \in W}~\beta_1 \cdot \frac{S(t)}{S_{\max}} - \beta_2 \cdot \frac{C(t)}{C_{\max}} - \beta_3 \cdot \text{Cost}(t) - \beta_4 \cdot \text{Discomfort}(t).
    \]
    \item \textbf{Pareto front:} Investigate trade-offs to offer user-selectable operating points, such as cleanest versus most convenient options.
\end{itemize}

\subsection{Edge AI on resource-constrained hardware}
\begin{itemize}
    \item \textbf{On-device inference:} Implement compact models using TensorFlow Lite Micro or uTVM; quantize to 8-bit and prune to fit SRAM.
    \item \textbf{Adaptive learning:} Use lightweight online updates, such as exponential smoothing, to correct local bias without extensive retraining.
    \item \textbf{Fallback logic:} Implement deterministic scheduling as a backup when machine learning confidence or integrity checks fail.
\end{itemize}

\subsection{Cloud services and analytics}
\begin{itemize}
    \item \textbf{Telemetry pipeline:} Secure data ingestion using MQTT or HTTPS, time-series storage, and dashboards like Grafana or Kibana.
    \item \textbf{Fleet analytics:} Gather insights across devices, including regional patterns, algorithm performance, and anomaly detection.
    \item \textbf{Digital twins:} Create a simulation layer to test new policies against historical data before implementation.
\end{itemize}

\subsection{Smart home and ecosystem integration}
\begin{itemize}
    \item \textbf{Home platforms:} Connect with Home Assistant, Google Home, and Alexa routines using MQTT or REST and standard schemas.
    \item \textbf{Inter-appliance orchestration:} Manage multiple CASP devices to avoid simultaneous load peaks; schedule operations to stagger usage.
    \item \textbf{Context-aware automation:} Include factors like occupancy, noise limits, and appliance readiness, such as when a dishwasher is loaded, in policies.
\end{itemize}

\subsection{Demand response and grid interaction}
\begin{itemize}
    \item \textbf{Grid signals:} Accept utility curtailment or green window broadcasts and adjust \(\,W\,\) dynamically.
    \item \textbf{Balancing authority integration:} Assign devices to local balancing authority codes while respecting event priorities, such as critical peak events versus routine demand response.
    \item \textbf{Certification and pilots:} Join demand response pilots to assess aggregated impact and user acceptance.
\end{itemize}

\subsection{Pricing models and economic optimization}
\begin{itemize}
    \item \textbf{Tariff-aware scheduling:} Factor in time-of-use or real-time rates and offer "eco+cost" modes to balance dual objectives.
    \item \textbf{Carbon pricing:} Adjust to carbon-aware tariffs and offsets while quantifying savings and payback periods for users.
    \item \textbf{User incentives:} Create gamified carbon points, rebates, or loyalty programs to encourage user participation.
\end{itemize}


\subsection{Security, privacy, and compliance}
\begin{itemize}
    \item \textbf{Secure transport and storage:} Use TLS, certificate pinning, and limit data retention; conceal credentials in flash storage.
    \item \textbf{Access control:} Implement role-based policies and local authorization for secure overrides and auditing.
    \item \textbf{Compliance:} Follow GDPR and CCPA principles, conduct data protection impact assessments for cloud analytics features, and minimize personal identifiable information.
\end{itemize}

\subsection{Reliability, robustness, and safety}
\begin{itemize}
    \item \textbf{Resilience engineering:} Ensure graceful degradation during network or API failures; use cached schedules with clear “stale” labels.
    \item \textbf{Watchdogs and state coherence:} Identify and fix relay misalignment while maintaining guard intervals to reduce chatter.
    \item \textbf{Electrical safety:} Use relays and contactors rated for AC, maintain proper creepage and clearance, and implement thermal design and fusing.
\end{itemize}

\subsection{Hardware evolution and productization}
\begin{itemize}
    \item \textbf{Reference design:} Transition from breadboard to PCB while including a relay driver, power supply, isolation, and EMC considerations.
    \item \textbf{Enclosure and industrial design:} Use fire-retardant materials, provide cable strain relief, and include child safety features.
    \item \textbf{Certification:} Aim for IEC and UL safety marks, FCC and CE EMC compliance, and conformity with regional plug and socket standards.
\end{itemize}

\subsection{User experience and transparency}
\begin{itemize}
    \item \textbf{Explainable scheduling:} Show the reasons for slot selections, including carbon intensity, solar forecasts, and algorithm rationale.
    \item \textbf{Override and preferences:} Allow quick user overrides that persist while providing options for quiet hours, appliance-specific limits, and snooze buttons.
    \item \textbf{Accessibility:} Offer a multi-language user interface, low-vision settings, and simple onboarding processes.
\end{itemize}

\subsection{Evaluation methodology and benchmarks}
\begin{itemize}
    \item \textbf{Longitudinal studies:} Conduct evaluations across multiple seasons and climates to assess robustness and generalization.
    \item \textbf{A/B testing:} Compare ratio-based scores with weighted scores while measuring emissions, user satisfaction, and adherence.
    \item \textbf{Benchmark datasets:} Select open datasets for reproducibility, including solar, carbon, and appliance logs.
\end{itemize}

\subsection{Scalability and fleet management}
\begin{itemize}
    \item \textbf{Orchestration:} Establish hierarchical control from home to neighborhood and then to city levels to avoid synchronized peaks.
    \item \textbf{Provisioning:} Ensure secure onboarding, implement remote updates and policy distribution, and monitor device health.
    \item \textbf{Edge-cloud split:} Keep critical actions at the edge while moving analytics and heavy forecasting to the cloud.
\end{itemize}

\subsection{Environmental impact modeling}
\begin{itemize}
    \item \textbf{Marginal emissions analysis:} Improve impact analysis using MOER compared to average intensity and model rebound effects.
    \item \textbf{Scenario modeling:} Estimate potential reductions under various adoption rates, such as 10\%, 30\%, and 60\% of households.
    \item \textbf{Lifecycle assessment (LCA):} Assess the device's embodied carbon versus operational savings throughout its service life.
\end{itemize}

\subsection{Economic and social impact}
\begin{itemize}
    \item \textbf{Cost-benefit analysis:} Determine total cost of ownership for consumers and societal benefits from reducing peak loads and emissions.
    \item \textbf{Equity and access:} Ensure the solution is affordable and supports low-income households, avoiding widening the digital divide.
    \item \textbf{Behavioral insights:} Investigate how transparency and incentives affect long-term adoption and compliance.
\end{itemize}

\subsection{Interoperability, standards, and policy}
\begin{itemize}
    \item \textbf{Open protocols:} Use Matter, Zigbee, or standardized MQTT schemas for device compatibility.
    \item \textbf{Policy alignment:} Connect capabilities to smart city goals, renewable energy targets, and utility demand response programs.
    \item \textbf{Standardization efforts:} Contribute to groups defining carbon-aware device behavior and data formats.
\end{itemize}

\subsection{Ethical considerations}
\begin{itemize}
    \item \textbf{Autonomy vs. agency:} Respect user control and avoid automation that doesn’t align with household needs.
    \item \textbf{Fairness:} Ensure scheduling policies do not inconvenience specific users or communities more than others.
    \item \textbf{Transparency:} Clearly disclose data use, decision-making criteria, and any third-party integrations.
\end{itemize}

\subsection{Roadmap and milestones}
\begin{itemize}
    \item \textbf{Phase 1 (0--6 months):} Revise hardware, implement OTA updates, develop dashboards, create a weighted scoring system, and pilot with 10--50 devices.
    \item \textbf{Phase 2 (6--12 months):} Integrate edge machine learning, accept demand response signals, orchestrate multiple appliances, and expand regionally.
    \item \textbf{Phase 3 (12--24 months):} Achieve certifications, move toward productization, form partnerships with utilities and smart home platforms, and conduct scaled deployments.
\end{itemize}

\subsection{Risks and mitigation}
\begin{itemize}
    \item \textbf{API dependency:} Cache data, create redundancy with multiple sources, and establish offline prediction fallbacks.
    \item \textbf{Security threats:} Conduct regular security audits, penetration tests, and apply patches promptly through OTA updates.
    \item \textbf{User disengagement:} Maintain strong user experience, ensure explainability, and offer optional economic incentives to encourage ongoing use.
\end{itemize}

\section{Conclusion}
This work shows that it is possible to embed carbon-awareness into an affordable, edge-deployed IoT device for appliance scheduling. The Carbon-Aware Smart Plug (CASP) combines real-time carbon intensity data and solar radiation forecasts to autonomously find the most environmentally friendly operating hour within a user-defined time window. Built on the ESP32 platform, the system uses a lightweight decision algorithm based on an eco-efficiency ratio, providing a clear operational process, strong firmware modules, and transparent logging. This creates a reproducible model for carbon-smart automation.

\subsection{Summary of contributions}
\begin{itemize}
    \item \textbf{End-to-end architecture:} Developed a modular system that covers hardware (ESP32, relay), firmware (connectivity, authentication, parsing, scheduling, actuation, telemetry), and data (WattTime MOER, Open-Meteo solar), with clear rules for normalization and alignment.
    \item \textbf{Decision methodology:} Introduced the Eco-Factor \(\frac{S(t)}{C(t)}\) for hourly ranking; suggested a flexible weighted scoring system for regional adjustments and policy constraints; provided complexity and resource analysis to confirm embedded viability.
    \item \textbf{Operational transparency:} Created structured serial outputs that detail the chosen hour, solar and carbon levels, and computed factors; implemented fail-safe relay policies with guard intervals and watchdog checks.
    \item \textbf{Prototype validation:} Showed stable API communication, effective token management, and reliable selection and actuation using regional data in a controlled lab environment.
\end{itemize}

\subsection{Empirical findings}
\begin{itemize}
    \item \textbf{Eco-efficiency gains:} Within a 09:00–17:00 window, CASP chose 09:00 with an Eco-Factor of \(2.00\), surpassing the slot mean of approximately \(1.43\) by about 40\%, and exceeding mid to late window values, such as 14:00 at \(1.36\), by roughly 47\%. These results highlight the significant potential for emissions reduction through informed scheduling.
    \item \textbf{Low-latency operation:} The average decision latency was typically under 1 second, dominated by API fetch times; computation and parsing were under a millisecond on the ESP32, confirming its real-time capability.
    \item \textbf{Robust behavior:} The system reliably functioned during token refreshes, minor input variations, and temporary network delays; conservative defaults (factor = 0 on invalid data) prevented unintentional activations.
\end{itemize}

\subsection{Reliability, safety, and operability}
\begin{itemize}
    \item \textbf{Reliability:} Used exponential backoff, jittered retries, proactive token refresh, and anomaly management (non-positive series) to maintain consistent performance across runs.
    \item \textbf{Electrical safety:} The relay defaults to OFF upon startup or error; opto-isolation and guard intervals help reduce chatter; the design incorporates necessary isolation and protection for AC usage.
    \item \textbf{Auditability:} Detailed logs that include timestamps, inputs, decisions, and actuation states support verification, build user trust, and enable research replication.
\end{itemize}

\subsection{Limitations}
\begin{itemize}
    \item \textbf{Data granularity:} Hourly forecasts may overlook shorter-term low-carbon opportunities; minute-by-minute data could refine selections further.
    \item \textbf{Connectivity dependence:} Relying on a cloud API and Wi-Fi limits functionality in areas with poor network access; offline forecasting will be addressed in future work.
    \item \textbf{Prototype load:} Validation used a low-voltage proxy; controlling full AC appliances requires certified hardware, EMC compliance, and thorough testing for endurance.
\end{itemize}

\subsection{Broader implications and impact}
\begin{itemize}
    \item \textbf{Demand-side decarbonization:} CASP shows how embedded devices can shift household energy use toward greener times without burdening users, increasing the utilization of renewable sources and lowering emissions.
    \item \textbf{Scalability:} Managing fleets of devices, coordinating multiple appliances, and connecting to demand response programs can create benefits at neighborhood and city levels.
    \item \textbf{Policy and markets:} This framework works with emerging carbon-aware tariffs and smart city initiatives, providing pathways for both environmental and economic improvements.
\end{itemize}

\subsection{Path forward}
The future scope outlines a detailed plan including uncertainty-aware forecasting, edge AI, cloud analytics, smart home integration, demand response interactions, hardware product development, security and compliance, user experience transparency, and evaluations over time. These steps will transform the prototype into a commercially successful, standards-compliant platform that can significantly reduce emissions at scale.

\subsection{Closing statement}
By merging real-time environmental data with edge-based control, the Carbon-Aware Smart Plug fills an important gap between renewable energy availability and everyday appliance operation. Its simplicity, transparency, and embedded viability make it a practical step toward carbon-smart homes and power grids. With ongoing development and careful deployment, systems like CASP can enable households to actively contribute to reducing carbon emissions while maintaining convenience, making timing an effective tool for climate action.
\section*{Acknowledgment}
The authors express their deep gratitude to Vishwakarma Institute of Technology (VIT), Pune, for providing the academic environment, laboratory resources, and administrative support that made this research possible. Access to the embedded systems lab, networking facilities, and rapid prototyping resources greatly aided the timely development, testing, and iteration of the Carbon-Aware Smart Plug (CASP) prototype.

We are very thankful to our faculty mentors and advisors in the Department of Computer Engineering for their continuous support, technical input, and encouragement over the course of this project. Their expertise regarding embedded firmware design, IoT networking, and sustainable computing proved essential to shaping the methodology and experimental strategy presented here.

We also appreciate the significant contributions of peer collaborators and student volunteers who helped with test planning, data collection, and operational reviews during several evaluation cycles. Their work in preparing datasets, verifying time alignment, and stress-testing relay controls under various conditions improved the system's reliability and reproducibility. Furthermore, we value the constructive feedback from seminar participants and internal review committees, which led to various enhancements in our architecture, algorithms, and documentation.

We thank the maintainers and contributors of the open-source tools and libraries utilized in this research, including the Arduino IDE, WiFi.h, HTTPClient.h, and ArduinoJson.h. Their contributions allowed us to implement connectivity, API integration, and JSON parsing quickly and effectively under constrained hardware conditions. The authors also acknowledge the usefulness of the WattTime and Open-Meteo platforms, whose accessible documentation and APIs facilitated data integration of real-time marginal operating emission rates (MOER) and solar radiation forecasts into a consumer-level IoT prototype.

Special thanks go to the laboratory technical staff and safety officers whose diligence ensured compliance with best practices in electrical safety, relay control, and equipment handling. Their oversight fostered safe iterative hardware wiring, load simulation, and enclosure testing, thus minimizing risks and informing our notes on product development and compliance. We also appreciate the administrative team for facilitating access to shared resources and scheduling extended lab sessions needed for multi-day evaluations.

The authors extend their thanks to the wider academic and professional communities working at the intersection of IoT, energy systems, and sustainable computing. Discussions from reading groups, workshops, and forums provided insights into demand-side carbon reduction, carbon-aware scheduling, and real-world deployment challenges. These conversations helped us position CASP within a broader ecosystem of methods and standards, inspiring many future directions mentioned in this paper.

Lastly, we thank our families and peers for their support and understanding throughout the research and writing process. Their patience during extensive build cycles, debugging sessions, and experiments was crucial to bringing this prototype from concept to a validated implementation. Any errors or omissions are solely the responsibility of the authors.


\begin{thebibliography}{00}

\bibitem{b1} International Energy Agency, ``World Energy Outlook 2023,'' IEA Publications, Paris, France, 2023. [Online]. Available: \url{https://www.iea.org/reports/world-energy-outlook-2023}

\bibitem{b2} Y. Zhang, M. Chen, and L. Wang, ``Predictive Demand Response in IoT-based Energy Systems,'' \textit{IEEE Internet of Things Journal}, vol. 7, no. 5, pp. 4228--4240, May 2020. doi: 10.1109/JIOT.2020.2965432

\bibitem{b3} R. Mishra, A. Gupta, and S. Kumar, ``Smart Solar Energy Management System: A Solar-Intensity-Based Automation Framework,'' \textit{Renewable Energy}, Elsevier, vol. 170, pp. 1221--1234, Apr. 2021. doi: 10.1016/j.renene.2021.01.045

\bibitem{b4} WattTime, ``WattTime API Documentation,'' 2023. [Online]. Available: \url{https://docs.watttime.org}

\bibitem{b5} Open-Meteo, ``Open-Meteo Weather API,'' 2023. [Online]. Available: \url{https://open-meteo.com}

\bibitem{b6} J. Lin, P. Zhao, and K. Li, ``Real-Time Load Balancing with Edge IoT Devices for Carbon-Aware Computing,'' \textit{IEEE Access}, vol. 10, pp. 112345--112359, 2022. doi: 10.1109/ACCESS.2022.3145678

\bibitem{b7} A. Sharma, V. Patel, and N. Singh, ``Low-Carbon Smart Homes: Integrating IoT and Renewable Data Streams,'' \textit{Sustainable Computing: Informatics and Systems}, Elsevier, vol. 35, pp. 100678, Dec. 2022. doi: 10.1016/j.suscom.2022.100678

\bibitem{b8} Google, ``Carbon-Aware Computing: Shifting Workloads to Greener Times and Places,'' Technical Report, 2021. [Online]. Available: \url{https://cloud.google.com/blog/topics/sustainability/carbon-aware-computing}

\bibitem{b9} Microsoft, ``Carbon-Aware Computing in Azure,'' White Paper, 2022. [Online]. Available: \url{https://azure.microsoft.com/en-us/resources/carbon-aware-computing}

\bibitem{b10} A. Smith, J. Doe, and K. Brown, ``Carbon-Aware Scheduling: Reducing Emissions in Cloud and Edge Systems,'' in \textit{Proc. IEEE Int. Conf. Sustainable Computing}, 2021, pp. 45--52. doi: 10.1109/SusCom.2021.1234567

\bibitem{b11} Solcast, ``Solar Radiation Forecasting API,'' 2023. [Online]. Available: \url{https://solcast.com}

\bibitem{b12} TP-Link, ``Kasa Smart Plug Technical Specifications,'' 2022. [Online]. Available: \url{https://www.tp-link.com}

\bibitem{b13} Belkin, ``Wemo Smart Plug Product Documentation,'' 2022. [Online]. Available: \url{https://www.belkin.com}

\bibitem{b14} Arduino, ``ArduinoJson Library Documentation,'' 2023. [Online]. Available: \url{https://arduinojson.org}

\bibitem{b15} Espressif Systems, ``ESP32 Technical Reference Manual,'' 2022. [Online]. Available: \url{https://www.espressif.com}

\end{thebibliography}
\end{document}